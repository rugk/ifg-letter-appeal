% !TeX program = lualatex
% !BIB TS-program = biber
% !TeX encoding = UTF-8
% !TeX spellcheck = de_DE

\documentclass[
paper=a4,
parskip=full,
twoside
]{scrartcl}
\usepackage[
%DIN5008A, % load up-to-date DIN if it exists (TODO: or DIN5008B???)
backaddress=false,
firstfoot=true,
enlargefirstpage=true,
% fromalign=false,
]{scrletter}
\parskip3mm
\parindent0mm

% encoding only necessary if neither luatex or xetex
\usepackage{iftex}
\ifPDFTeX
\usepackage[T1]{fontenc}
\fi

% deutsch
\usepackage[
english,
main=ngerman,
% shorthands=off
]{babel}
\usepackage[ngerman]{isodate}\usepackage[autostyle, german=quotes]{csquotes} % deutsche Anführungszeichen mit \enquote

% better page numbers
% https://tex.stackexchange.com/a/578140/98645
\renewcommand*\pagemark{%
	\usekomafont{pagenumber}{\hypersetup{hidelinks}-~\thepage~von~\letterlastpage~-}%
}
\let\letterpagemark\pagemark
% https://tex.stackexchange.com/q/578016/98645
%\usepackage{scrlayer-scrpage}
%\cfoot{- \thepage\ von \letterlastpage\ -}
% also show page number on first page
% https://tex.stackexchange.com/a/578050/98645
\newcommand{\originalopening}{}
\let\originalopening\opening
\renewcommand{\opening}[1]{\originalopening{#1}\thispagestyle{plain}}
\KOMAoptions{firstfoot=false}% disable first footer

% Set Page layout:
%\usepackage{changepage}
%\changepage{text height}{text width}{even-side margin}
%{odd-side margin}{column sep.}
%{topmargin}{headheight}{headsep}{footskip}
%\changepage{+3cm}{}{}{}{}{}{}{}{-5cm}
\LoadLetterOption{DIN}
\LoadLetterOption{sender}

% LOAD ICONS
% symbols (look oldish)
\RequirePackage{marvosym}
% use \Telefon, \Letter, \Mobilefone
% better icons: font awesome
% e.g. \faMobile, \faMobile*, \faPhone, \faPhone*, \faEnvelope, \faEnvelope[regular]
\usepackage{fontawesome5}
% \usepackage{lipsum} % for testing
% for logos
%\usepackage{graphicx}

% layout
%\renewcommand{\normalsize}{\fontsize{12pt}{12pt}\selectfont} % change font size % TODO: better method?
% \renewcommand{\small}{\fontsize{10pt}{12pt}\selectfont}
% \renewcommand{\footnotesize}{\fontsize{8pt}{9.6pt}\selectfont}
\usepackage[
inner=2cm, % 2,5 cm - 3,0 cm (left)
outer=1.5cm, % 1,5 cm - 2,0 cm (right)
top=2.0cm, % 2,0 cm - 2,5 cm
bottom=1cm,
includefoot,
heightrounded
]{geometry}

% Farben und PDF-Eigenschaften
\usepackage[dvipsnames, table]{xcolor}
\usepackage{url}
\usepackage[
colorlinks=true,
% linktoc=all,
pdfusetitle,
]{hyperref}

\makeatletter
\hypersetup{
	pdfsubject={Widerspruch IFG},
}
\makeatother

\newcommand\myshade{85}
\definecolor{auburn}{rgb}{0.43, 0.21, 0.1}
\definecolor{cyanDark}{HTML}{155b44}
\colorlet{mylinkcolor}{auburn}
\colorlet{mycitecolor}{YellowOrange}
\colorlet{myurlcolor}{Aquamarine}
\colorlet{myfilecolor}{cyanDark}
\colorlet{tablealternate}{yellow!12}

\hypersetup{
	linkcolor  = mylinkcolor!\myshade!black,
	citecolor  = ., % mycitecolor!\myshade!black,
	urlcolor   = myfilecolor!100!black,
	filecolor  = myfilecolor!\myshade!black,
	colorlinks = true,
}

% Referenzen
%\usepackage{biblatex-german-legal}
\usepackage[style=biblatex-juradiss,
sortcites=true,
sorting=none,
defernumbers=true,
%maxcitenames=3,
minbibnames=3, % cite up to three authors in bib
backref=true,
giveninits=true,
loccittracker=constrict,
backend=biber]{biblatex}
\addbibresource{references/ifg.bib}

\DeclareLabeldate{%
	\field{date}
	\field{year}
	\field{eventdate}
	\field{origdate}
	\field{urldate}
	\literal{nodate}
}

\ExecuteBibliographyOptions{labeldateparts}

% defines \lawcite for inline law citations https://tex.stackexchange.com/a/592383/98645
%wrapper for the ibid citing
\DeclareCiteCommand{\legalcite}[\mkbibbrackets]
{\usebibmacro{prenote}}
{%
	\ifthenelse{\ifciteibid\AND\NOT\iffirstonpage}%
	{%
		\ifloccit{%
			\global\booltrue{cbx:loccit}%
			\usebibmacro{cite:ibid}%
		}{%
			\global\boolfalse{cbx:loccit}%
			\usebibmacro{cite:loccit}}}%
	{\usebibmacro{cite:legal}\usebibmacro{cite:save}}%
}{\multicitedelim}
{\usebibmacro{cite:postnote}}


%cite format
\newbibmacro*{cite:legal}{%
	\usebibmacro{citeindex}%
	\iffieldundef{shorthand}%
	{%
		\ifnameundef{labelname} {} {%
			\printnames[family-given]{labelname}%
			\setunit{\printdelim{nameyeardelim}}%
		}
		\iffieldundef{labelyear} {} {
			\usebibmacro{cite:labeldate+extradate}%
			\setunit{\addperiod\addspace}\printdelim{yeartitledelim}}
		\usebibmacro{cite:label}
		\setunit{\addperiod\addspace}
		\usebibmacro{cite:location+publisher}%
	}%
	{\usebibmacro{cite:shorthand}}%
}

%helper macros, this one is the definition of authoryear.cbx
\newbibmacro*{cite:label}{%
	\iffieldundef{label}
	{\printtext[bibhyperref]{\printfield[citetitle]{labeltitle}}}
	{\printtext[bibhyperref]{\printfield{label}}}}


\newbibmacro*{cite:labeldate+extradate}{%
	\iffieldundef{labelyear}
	{}
	{\printtext[parens]{\printlabeldateextra}}}

\newbibmacro*{cite:location+publisher}{%
	\iflistundef{location}{}{
		\printlist{location}%
		\iflistundef{publisher}
		{\setunit*{\addcomma\space}}
		{\setunit*{\addcolon\space}}%
	}%
	\printlist{publisher}%
	\setunit{\addperiod}%
	\newunit
}

%add loccite support, adopted from biblatex authortitle-ibid.cbx
%\newbibmacro*{cite:postnote}{%
%	\ifbool{cbx:loccit}%
%	{}%
%	{\usebibmacro{postnote}}%
%}

\newbibmacro*{cite:loccit}{%
	\printtext[bibhyperref]{\bibstring[\mkibid]{loccit}}}

\AtBeginDocument{%to be changed after language setup
	\renewcommand*{\multinamedelim}{\addcomma\addspace}
	\renewcommand*{\finalnamedelim}{\addspace\&\addspace}
}

%In case you also want to get ibid, with empty postnote but that's overriding the test
\makeatletter
\def\blx@loccit@numcheck#1{%
   \blx@imc@iffieldundef{postnote}%
   {%
       \ifcsundef{blx@lastnote@#1@\abx@field@entrykey}%
           {\@firstoftwo}%
           {\@secondoftwo}%
   }%
   {%
       \expandafter\blx@imc@ifpages%
       \expandafter{\abx@field@postnote}%
       {\blx@imc@iffieldequalcs{postnote}{blx@lastnote@#1@\abx@field@entrykey}}%
       {\@secondoftwo}}}%
\ExplSyntaxOff
\makeatother

% font
\usepackage[sfdefault]{roboto}
% Set Font: sans serif Latin Modern
%\usepackage{lmodern}
% \renewcommand*\familydefault{\sfdefault}

% Switch implementation
\usepackage{xifthen}
\newcommand{\ifequals}[2]{\ifthenelse{\equal{#1}{#2}}}
\newcommand{\case}[2]{#1 #2} % Dummy, so \renewcommand has something to overwrite...
\newenvironment{switch}[1]{\renewcommand{\case}{\ifequals{#1}}}{}

% Set Font: sans serif Latin Modern
%\usepackage{lmodern}
% \renewcommand*\familydefault{\sfdefault}

% mail merge  Serienbrief
\usepackage{ifthen,mailmerge}
% \ifequal{A}{B}{what if A=B}{what if A<>B}
\newcommand{\ifequal}[2]{\ifthenelse{\equal{#1}{#2}}}
% if not empty
\newcommand{\ifNonEmpty}[3]{\ifequal{#1}{}{#3}{#2}}
\newcommand{\ifField}[1]{\ifNonEmpty{\field{#1}}}
\newcommand{\ifFieldEqual}[1]{\ifequal{\field{#1}}}
% \newcommand{\fieldOptional}[2]{\ifDefined{\field{#1}}{\field{#1}}}

\usepackage[
letterspace=150, % https://tex.stackexchange.com/a/62351/98645
final
]{microtype} % better hyphenation / breaks % TODO: https://tex.stackexchange.com/questions/82001/microtype-settings-for-dummies
\microtypesetup{activate=true}

% \setlength{\parindent}{0pt}
% \setlength{\parskip}{\baselineskip}

\newcommand{\textsharpNew}{\#} %  https://tex.stackexchange.com/questions/44528/how-to-make-the-correct-hash-symbol-in-c-sharp-c

\newenvironment{enumerateCompact}
{ \begin{enumerate}
	\setlength{\sep}{0pt}
	\setlength{\parskip}{0pt}
	\setlength{\parsep}{0pt}     }
{ \end{enumerate}                  }
\newenvironment{itemizeCompact}
{ \begin{itemize}
	\setlength{\itemsep}{0pt}
	\setlength{\parskip}{0pt}
	\setlength{\parsep}{0pt}     }
{ \end{itemize}                  }

%
% style fixes
%

% change itemize labels (switches first and second level symbols)
\renewcommand{\labelitemi}{\normalfont\bfseries \textendash}
\renewcommand{\labelitemii}{\textbullet}

% fix footnotes, https://tex.stackexchange.com/a/40091/98645
\newcommand\fnsep{\textsuperscript{,}}

% don't split on pages https://tex.stackexchange.com/a/32210/98645
\interfootnotelinepenalty=10000

%%%%%% MAIL FIELDS
\mailfields{
	receiver,
	name,
	shortname,
	address,
	specialmail,
	anrede,
	anredeName,
	closing,
	appendixLabel,
	appendix,
	ps,
}
% NOTE: always add an empty field (i.e. comma ,) at the end!

\begin{document}
	% \def\today{1st January, 1895}

	% Set Appendix text at very end (double point will be set automatically)
	\setkomavar*{enclseparator}{Anhang}

	% FragDenStaat
	\newcommand{\requestTitle}{Informationsanfrage zu Beispieldaten}
	\newcommand{\requestNumber}{123456}
	\newcommand{\requestMail}{entermailadresshere@fragdenstaat.de}
	\newcommand{\requestUploadToken}{copytokenforuploadhere}
	\newcommand{\requestDate}{01.~Januar~2021}
	\newcommand{\requestNoticeDate}{01.~Februar~2021}
	\newcommand{\requestNoticeRef}{123456""{\textsharpNew}EXAMPLE""{\textsharpNew}0042}
	\newcommand{\mediationRef}{25-729/002~II{\textsharpNew}XXXX}

	\setkomavar{yourref}{\requestNoticeRef}
	\setkomavar{yourmail}{\requestNoticeDate}
	\renewcommand{\myrefname}{Mein Zeichen}
	\setkomavar{myref}{{\textsharpNew}\requestNumber}

	\mailrepeat{
		\begin{letter}{%
				\field{address}
			}

			% custom sender per receiver
			\LoadLetterOption{sender/me-\field{receiver}}
			\ifField{specialmail}{
				\setkomavar{specialmail}{\field{specialmail}}
			}

			% set PDF metadata
			\hypersetup{
				pdftitle={Widerspruch gegen den Bescheid vom \requestNoticeDate},
				pdfauthor={{\ownname}},
				pdfsubject={Auskunft nach dem IFG \textendash{} \requestTitle},
			}

			% subject, date, place:
			\setkomavar{subject}{%
				Betreff: Auskunft nach dem IFG \textendash{} \requestTitle\\
				Hier: Widerspruch gegen den Bescheid vom \requestNoticeDate
			}
			%\setkomavar{title}{Widerspruch IFG-Bescheid}

			\opening{
				\ifField{anrede}{\field{anrede}}{Sehr geehrte Damen und Herren}%
				\ifNonEmpty{anrede}{}{\ifField{anredeName}{\field{anredeName}}{\field{name}}}%
				,
			}

			% HowTo Fields
			% name: \ifField{name}{NichtLeer \field{name}}{Leer}\\
			% blaField: \ifField{blaField}{NichtLeer}{Leer}\\
			% leer: \ifNonEmpty{}{NIchtLeer}{Leer}\\
			% nicht leer: \ifNonEmpty{n}{NIchtLeer}{Leer}\\

			% Test
			% \ifthenelse{\equal{1}{1}}{True}{False}
			% \ifthenelse{\equal{du}{du}}{True}{False}
			% \ifthenelse{\equal{\field{anredeForm}}{}}{True}{False}

			%
			% Einleitung
			%

			gegen Ihren Bescheid vom
			\requestNoticeDate{}
			(Gz. \requestNoticeRef;
			Dok-Nr. \requestNoticeRef{\textsharpNew}0004)
			lege ich hiermit\\
			\begin{center}
				\vspace{-20pt}
				\huge
				\robotoMedium{\textls{WIDERSPRUCH}}
				\vspace{-10pt}
			\end{center}
			ein.

			%
			% Sachverhalt
			%

			\subsubsection*{Sachverhalt} \vspace{-10pt}

			% Einleitung
			Mittels E-Mail vom \requestDate{} beantragte ich auf Grundlage des Informationsfreiheitsgesetzes (IFG), des Umweltinformationsgesetzes (UIG) und des Gesetzes zur Verbesserung der gesundheitsbezogenenen Verbraucherinformation (VIG) die Zusendung des
			%
			\enquote{Verzeichnis von Verarbeitungstätigkeiten gemäß Artikel 30~DSGVO sowie §\,70~BDSG}.
			%
			Der genaue, vollständige Wortlaut soll hierbei nicht erneut wiederholt werden, da er beiden Parteien zugänglich ist.

			% Ablehnung Behörde
			In Ihrem
			elektronisch übermittelten
			Bescheid vom
			\requestNoticeDate{}
			%			(Dok-Nr. \requestNoticeRef""{\textsharpNew}0004)
			lehnen Sie als \field{name} (\field{shortname})
			mein Informationersuchen auf Basis meines Antrags
			%			nach erheblicher Verspätung
			ab.
			% Ablehnung Behörde Begründung
			Sie lehnen die Anfrage nach §§\,5 und 6~IFG sowie nach §~3 Nr.~1 ab.
			Als Grund für die zuerst genannte Rechtsgrundlage nennen Sie, dass Daten Ihrer Kunden betroffen seien.
			Auch hierbei wird aus obig bereits genannten Gründen auf eine Wiederholung des genauen Wortlauts verzichtet.

			%
			% Gegenargumentation
			%

			\subsubsection*{Begründung} \vspace{-10pt}
			% Zeitverzug
			Nach~§\,9 Abs.\,1~IFG ist die Bekanntgabe der Entscheidung bei Ablehnung oder teilweiser Ablehnung zwingend innerhalb eines Monats nach Antragstellung zu erfolgen. Dies \enquote{entspricht auch dem Zweck des Gesetzes, möglichst bald Klarheit über den Informationszugang oder dessen Verweigerung zu schaffen. Schließlich entspricht diese Auslegung dem Willen des Gesetzgebers}\legalcite[117]{djvIfgKommentierung}.
			\enquote{Im Ergebnis ist § 9 Abs. 1 danach so auszulegen,
				dass eine ablehnende Entscheidung \textelp{} zwingend
				binnen eines Monats bekannt zu geben ist. Das
				gilt für ablehnende oder teilweise ablehnende
				Bescheide.}
			Dies ist in dem hier vorliegenden Fall nicht geschehen. Aus diesem Grund erachte ich den Bescheid für unzulässig. Weitere Gründe sind dabei nicht ausgeschlossen.


			% § 3 IFG Allg.
			% Generell besteht ein Regel-Ausnahme-Verhältnis, \enquote{die Verweigerung des Zugangs die begründungspflichtige Ausnahme}\legalcite{djvIfgKommentierung} darstellt. Auch sind Ausnahmetatbestände der §§\,3 bis 6 \enquote{nach den üblichen Auslegungsregeln eng zu verstehen}\legalcite{djvIfgKommentierung}.
			Generell gilt, dass
			\enquote{in der Begründung einer ablehnenden Entscheidung \textelp{} die nachteiligen Auswirkungen konkret zu benennen und \textendash darzulegen}\legalcite{djvIfgKommentierung} ist,
			\enquote{warum die Möglichkeit besteht, dass solche \textins{nachteilige} Auswirkungen \textins{nach §\,3 IFG Nr.~1} eintreten}\legalcite{djvIfgKommentierung}.
			Es ist erforderlich \enquote{dass die Möglichkeit, dass das jeweilige Schutzgut beeinträchtigt wird, besteht und von der ablehnenden Behörde auch dargelegt wird}\legalcite{djvIfgKommentierung}.

			% § 4
			Eine Ablehnung nach §\,4 IFG scheidet deswegen aus, weil kein laufender Vorgang oder Entscheidungsverfahren existiert, welche den \enquote{Erfolg der Entscheidung oder bevorstehender behördlicher Maßnahmen vereitelt würde}, da es sich um ein feststehendes einmalig in der Vergangenheit Dokument handelt, welches meiner Einschätzung nach nur selten aktualisiert wird.

			% Verschlusssache
			Sollten Teile der Informationen als Verschlusssache eingestuft sein, so kann sich eine \enquote{Ablehnung eines Informationsantrages, die auf §\,3 Nr.\,4 gestützt werden soll, \textelp{} nicht auf den formellen Hinweis, es liege beispielsweise eine Verschlusssache vor, beschränken.}\legalcite[84][65]{djvIfgKommentierung}

			%
			% Internal
			%

			% Frist
			% https://forum.okfn.de/t/muss-kann-man-einen-widerspruch-befristen/1091?u=rugk
			Bzgl. der Fristen dieses Widerspruchs verweise ich auf die allgemein gültigen Rechtsbestimmungen, insbesondere
			% Zustellung Verwaltungsrecht allgemein
			§§\,41 Abs.\,1, 5 VwVfG, 3 Abs.\,1, 2 S.\,1 VwZG,
			% Briefzustellung
			%§\,180 S.\,1 und 2 ZPO,
			% Urlaubsfristen
			§\,57 Abs.\,2 VwGO i.V.m. §\,222 Abs.\,2 ZPO,
			% BGB Fristberechung allg. (Fristbeginn + Fristende)
			§§\,187 Abs.\,1, 188 Abs.\,2 BGB,
			§\,79 VwVfG,
			% allgemein
			%§\,57 Abs.\,2 VwGO,
			§\,222 Abs.\,2 ZPO
			sowie
			% Zustellung elektronisch
			§\,41\,VwVfG Abs.\,2 S.\,2, wonach \enquote{\textins*{e}in Verwaltungsakt, der im Inland oder in das Ausland elektronisch übermittelt wird, \textelp{} am dritten Tag nach der Absendung als bekannt gegeben \textins{gilt}}.

			%
			% Schluss
			%

			% Bitte Abwarten Vermittlung
			Ich möchte Sie ferner bitten, mein gestartetes Vermittlungsverfahren nach §\,12~IFG beim Bundesbeauftragten für den Datenschutz und die Informationsfreiheit~(BfDI) zu diesem Fall%
			, betroffenes Aktenzeichen
			\mediationRef,
			vor Bescheidung des Widerspruchs abzuwarten und die fachkundige rechtliche Beurteilung dieser unabhängigen dritten Stelle mit einfließen zu lassen.

			%\lipsum\lipsum  % for testing

			\closing{\ifField{closing}{\field{closing}}{Mit freundlichen Grü{\ss}en}}

			\ifField{appendixLabel}{
				\setkomavar*{enclseparator}{\field{appendixLabel}}
			}{
				% workaround, because we cannot disable adding the appendix conditionally
				% because we would have to include the appendix in an env (\ifField)
				\setkomavar*{enclseparator}{}
			}\
			%    \ifField{appendix}{}{\setkomavar*{enclseparator}{}} % fails
			\encl{\field{appendix}}

			\ifField{ps}{
				\ps PS:~{\field{ps}}
			}{}

			\par%
			Anfragenr: \requestNumber \\
			Antwort an: \usekomavar{fromemail} \\
			Laden Sie gro{\ss}e Dateien zu dieser Anfrage hier hoch:\\
			\url{https://fragdenstaat.de/anfrage/\requestNumber/upload/\requestUploadToken/}

			% FragDenStaat footer
			%--- \\
			\par\noindent\rule{0.2\textwidth}{0.4pt}\\
			%\footnoterule
			Hinweis: Ihre Antwort wird von mir ggf. auf der Plattform FragDenStaat.de veröffentlicht. Sämtliche personenbezogene Daten werde ich selbstverständlich unkenntlich machen.\\
			Falls Sie Fragen dazu haben oder eine Idee, was für eine Anfrage bei Ihnen im Haus notwendig wäre, besuchen Sie:
			\url{https://fragdenstaat.de/hilfe/fuer-behoerden/}
	\end{letter}}

	\mailentry{%
	{agency},% receiver
	{Informationstechnikzentrum Bund},% name
	{ITZBund},% shortname
	{Informationstechnikzentrum Bund\\Bernkasteler Stra{\ss}e~8\\53175~Bonn},% address,
	{},% specialmail
	{},% anrede
	{},% anredeName
	{},% closing
	{},% appendixLabel
	{},% appendix
	{},% }ps
}

\end{document}
